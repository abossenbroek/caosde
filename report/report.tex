\documentclass[a4paper,onecolumn]{IEEEtran}
\usepackage{amsmath}
\usepackage[english]{babel}

\author{A. Bossenbroek}
\title{Lab Report Computational Aspects of Stochastic Differential Equations}

\begin{document}
\maketitle
Assuming the volatility to be constant, as is done in the seminal work by Black
and Scholes, can be too much of a simplification. In this work a stock process
is considered which has a drift term, volatility, modeled as a mean reverting
stochastic process. The stock process is given as:
\newcommand{\indst}{\ensuremath{^{(1)}}}
\newcommand{\indvl}{\ensuremath{^{(2)}}}
\begin{equation}\label{eq:stock}
dS_t = \mu S_t dt + \sigma_t S_t dW_t\indst
\end{equation}
the volatility of the stock is defined to be:
\begin{equation}\label{eq:vol}
d\sigma_t = -(\sigma_t - \xi_t)dt + p\sigma_t dW_t\indvl
\end{equation}
which includes a mean reverting term $\xi_t$ defined as follows:
\begin{equation}\label{eq:mnrvrt}
\xi_t = \frac{1}{\alpha}(\sigma_t - \xi_t)dt
\end{equation}
The Brownian motions $W_t^{\{(1), (2)\}}$ in  equation \eqref{eq:stock} and
\eqref{eq:vol} are two independent identically distributed random variables.
They have the property that $W_0=0$ and $W_t - W_s \sim N(0, t - s)$ for $s <
t$.

\section{Numerical Methods}
\subsection{Stock Process}
To solve the stochastic differential equation (SDE) which models the stock 
process numerically three numerical methods are used. All methods rely on
rewriting the SDE in equation \eqref{eq:stock} to an It\=o process. For this two
auxiliary functions are introduced 
\begin{align}\label{eq:abstck}
a(S_t)&=\mu S_t&b(S_t)&=\sigma_t S_t.
\end{align}
Note that both functions do not include the independent variable $t$.  The
stock process can then be rewritten to:
\newcommand{\inttim}{\ensuremath{\int_{t_0}^{t}}}
\newcommand{\intdel}{\ensuremath{\int_{t}^{t + \Delta t}}}
\begin{equation*}
dS_t = a(S_t)dS_t + b(S_t)dW_t\indst
\end{equation*}
or
\begin{equation*}
S_t = S_0 + \inttim a(S_t)ds + \inttim b(S_t)dW\indst.
\end{equation*}
This value can only be approximated by taking a --infinitesimal-- small step
$\Delta t$ from $t$. The new value $S_{t + \Delta t}$ can be found using
different numerical methods. Three methods are discussed below.

\subsubsection{Euler Method}
The Euler method relies on computing the value of $a$ and $b$ as defined in
equation \eqref{eq:abstck}. For an It\=o process the Euler method is defined as:
\begin{equation}\label{eq:euler}
X_{t + \Delta t} = X_t + a\Delta t + b\Delta W
\end{equation}
where $\Delta t$ is the time step more commonly denoted as $h$ in numerical
mathematics and $\Delta W = \left(W\indst_{t + \Delta t} - W\indst_t\right)$.
Since $\left(W\indst_{t + \Delta t} - W\indst_t\right) =
Z\indst\sqrt{\Delta t}$ where $Z\indst\sim N(0, 1)$ the numerical
approximation can easily be found using the Euler method. Given a step size
$\Delta t$ the new stock value can be computed as:
\begin{align}
S_{t + \Delta t} = S_t + \mu S_t\Delta t + \sigma_t S_t Z\indst\sqrt{\Delta t}
\end{align}
where only $a(S_t)$ and $b(S_t)$ have to be computed, this in contrast to
subsequent methods.


\subsubsection{Milstein Method}
The Milstein method is a second method which can be used to compute the
numerical approximation of a SDE. The functions $a$ and $b$ as defined in
equation \eqref{eq:abstck} are used. In conjunction with these functions, the
Milstein method necessitates also $b'(S_t) = \sigma_t$. For an It\=o process
$X_t$ the Milstein method is defined as:
\begin{equation}\label{eq:mil}
X_{t + \Delta t} = X_t + a\Delta t + b\Delta W +
\frac{1}{2}bb'\left(\left(\Delta W\right)^2 - \Delta t\right)
\end{equation}
with $\Delta t$ and $\Delta W$ as defined before. For the SDE in
equation \eqref{eq:stock} the approximation is:
\begin{equation}
S_{t + \Delta t} = S_t + \mu S_t dt + \sigma_t S_t Z\indst\sqrt{\Delta t} +
	\frac{1}{2} \sigma_t^2 S_t \left(\left(Z\indst\sqrt{\Delta t}\right)^2 -
\Delta t\right)
\end{equation}
In contrast to the Euler method, the Milstein method not only requires to
compute $a(S_t)$ and $b(S_t)$, it also requires $b'(S_t)$.

\subsubsection{Runge-Kutta Method}
For a numerical approximation of a SDE using the Milstein method the
evaluation of $b'$ is necessary. However, in some circumstances $b'$ may be
unavailable or too expensive to compute. Therefore another numerical method
is also used. The Runge-Kutta method only necessitates the evaluation of $a$
and $b$ which similar to the Euler method. For an It\=o process $X_t$ the
Runge-Kutta method is defined as
\begin{equation}\label{eq:rk}
X_{t + \Delta t} = X_t + a\Delta t + b\Delta W + \frac{1}{2\sqrt{\Delta
t}}\left(\left(\Delta W\right)^2 - \Delta
t\right)[b\left(\bar{X_t}\right)-b(X_t)]
\end{equation}
where $\bar{X_t}$ is defined as:
\begin{equation}
\bar{X_t} = X_t + a\Delta t + b \sqrt{\Delta t}
\end{equation}
Applying this method to the stock process defined in equation \eqref{eq:stock}
the Runge-Kutta approximation can be computed as:
\begin{equation}
\begin{split}
\bar{S_t} &= S_t + \mu S_t\Delta t + \sigma_t S_t \sqrt{\Delta t}\\
S_{t + \Delta t} &= S_t + \mu S_t \Delta t + \sigma_t S_t Z\indst\sqrt{\Delta
t} + \frac{1}{2\sqrt{\Delta t}}\left(\left(Z\indst\sqrt{\Delta t}\right)^2 -
\Delta t\right)[\sigma_t\bar{S_t} - \sigma_t S_t]
\end{split}
\end{equation}
using this method the computational times can be decreased.

\subsection{Volatility Process}
In the previous section the numerical approximation of the stock process as
defined in equation \eqref{eq:stock} was discussed. In this section we will
discuss how the numerical approximation of the volatility process as defined
in equation \eqref{eq:vol} can be approximated. As with the stock process three
different numerical methods will be used. Since we already defined the Euler
method (see equation \eqref{eq:euler}), Milstein method (see equation
\eqref{eq:mil}) and Runge-Kutta method (see equation \eqref{eq:rk}) for an
It\=o process, this section will only discuss the applications of these
methods on the volatility process. For each approximation method we define $a$
and $b$ as follows:
\begin{align}
a(\sigma_t)&= -(\sigma_t - \xi_t) & b(\sigma_t) = p\sigma_t
\end{align}
The approximation of the volatility process using the Euler method can be
found with:
\begin{equation}
\sigma_{t + \Delta t} = \sigma_t - (\sigma_t - \xi_t)\Delta t +
	p\sigma_tZ\indvl\sqrt{\Delta t}
\end{equation}
where $Z\indvl\sim N(0, 1)$ is independent from $Z\indst$.

The approximation of the volatility process using the Milstein method can be
found with:
\begin{equation}
\sigma_{t + \Delta t} = \sigma_t - (\sigma_t - \xi_t)\Delta t +
	p\sigma_t Z\indvl \sqrt{\Delta t} + \frac{1}{2}p^2\sigma_t 
	\left(\left(Z\indvl\sqrt{\Delta t}\right)^2 - \Delta t\right)
\end{equation}

The Runge-Kutta method for the approximation of the volatility process is:
\begin{equation}
\begin{split}
\bar{\sigma_t} &= \sigma_t  -(\sigma_t - \xi_t)\Delta t + p\sigma_t 
	\sqrt{\Delta t}\\
\sigma_{t + \Delta t} &= \sigma_t - (\sigma_t - \xi_t)\Delta t 
	+ p\sigma_t Z\indvl\sqrt{\Delta t}
	+ \frac{1}{2\sqrt{\Delta t}}\left(\left(Z\indvl\sqrt{\Delta t}\right)^2 -
\Delta t\right)[p\bar{\sigma_t} - p\sigma_t]
\end{split}
\end{equation}

\subsection{Mean-Reverting Term}
A feature of the volatility process defined in equation \eqref{eq:vol} is that
it is mean reverting. The definition of the mean reverting term is given in
\eqref{eq:mnrvrt}. Since this function is deterministic the previous
stochastic approximation methods cannot be used. However, the numerical
approximation is essential. In subsequent sections the approximation error of
each individual stochastic numerical approximation will be the center of our
investigation. To reduce the approximation error originating from the
numerical integration of the mean reverting process a fourth-order
approximation method is used.  The Runge-Kutta numerical approximation method
(from this point forward revert to as the RK4) for the ordinary differential
equation defined in equation \eqref{eq:mnrvrt} is:
\begin{equation}
\xi_{t + \Delta t} = \xi_t + \frac{\Delta t}{6}(k_1 + 2k_2 + 2k_3 + k_4)
\end{equation}
where $k_{\{1, 2, 3, 4\}}$ are defined as:
\begin{align*}
k_1 &= \frac{1}{\alpha}(\sigma_t - \xi_t)\\
k_2 &= \frac{1}{\alpha}\left(\left(\sigma_t + \frac{\Delta t}{2}k_1\right) -
	\xi_t\right)\\
k_3 &= \frac{1}{\alpha}\left(\left(\sigma_t + \frac{\Delta t}{2}k_2\right) -
	\xi_t\right)\\
k_4 &= \frac{1}{\alpha}\left(\left(\sigma_t + \Delta t k_3\right) -
	\xi_t\right)
\end{align*}
With this last numerical approximation it is possible to implement the methods
in an algorithm and find solutions.

\section{Experiments}
Various experiments were conducted by means of computer programs which
implement the numerical approximations methods discussed in the previous
section. The first experiments investigate the impact of $p$ and $\alpha$ on
the behaviour of $S_{0 \leq t < T}$, $E(S_{0 \leq t < T})$ and $E(S_T)$.


\end{document}

% vim: spell spelllang=en
